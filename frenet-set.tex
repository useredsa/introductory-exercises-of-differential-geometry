%%%%%%%%%%%%%%%%%%%%%%%%%%%%%%%%%%%%%%%%%
% fphw Assignment
% LaTeX Template
% Version 1.0 (27/04/2019)
%
% This template originates from:
% https://www.LaTeXTemplates.com
%
% Authors:
% Class by Felipe Portales-Oliva (f.portales.oliva@gmail.com) with template 
% content and modifications by Vel (vel@LaTeXTemplates.com)
%
% Template (this file) License:
% CC BY-NC-SA 3.0 (http://creativecommons.org/licenses/by-nc-sa/3.0/)
%
%%%%%%%%%%%%%%%%%%%%%%%%%%%%%%%%%%%%%%%%%

%----------------------------------------------------------------------------------------
%	PACKAGES AND OTHER DOCUMENT CONFIGURATIONS
%----------------------------------------------------------------------------------------

\documentclass[
	12pt, % Default font size, values between 10pt-12pt are allowed
	%letterpaper, % Uncomment for US letter paper size
	%spanish, % Uncomment for Spanish
]{fphw}

% Template-specific packages
\usepackage[T1]{fontenc} % Output font encoding for international characters
\usepackage{fontspec,unicode-math} % Required for using utf8 characters in math mode
\usepackage{parskip}  % To add extra space between paragraphs
\usepackage{graphicx} % Required for including images
\usepackage{booktabs} % Required for better horizontal rules in tables
\usepackage{enumerate}% To modify the enumerate environment
\usepackage{mathtools}% Important mathematicals environments and other things
\usepackage{physics}  % Defines lots of commands for vectors, matrices, derivatives...
\setlength{\parindent}{15pt}
\setlength{\headheight}{22.66pt}

%----------------------------------------------------------------------------------------
%	ASSIGNMENT INFORMATION
%----------------------------------------------------------------------------------------

\title{Task 3 \\ Frenet's Family} % Assignment title

\author{Emilio Domínguez Sánchez} % Student name

\date{October 22nd, 2020} % Due date

\institute{University of Murcia \\ Faculty of Mathematics} % Institute or school name

\class{Geometría de Superficies} % Course or class name

\professor{Dr. Pascual Lucas Saorin} % Professor or teacher in charge of the assignment

%----------------------------------------------------------------------------------------
%	Definitions
%----------------------------------------------------------------------------------------

\DeclareMathOperator{\proy}{proy}
\newcommand{\R}{\mathbb{R}}
\newcommand{\T}{{\vb{T}}}
\newcommand{\N}{{\vb{N}}}
\newcommand{\B}{{\vb{B}}}
\newcommand{\α}{{\vb*{α}}}
\newcommand{\inner}[2]{\left\langle #1, \; #2 \right\rangle}
\renewcommand{\dv}{\prime}
\newcommand{\ddv}{\dprime}
\newcommand{\dddv}{\trprime}

\begin{document}

\maketitle % Output the assignment title, created automatically using the information in the custom commands above

%----------------------------------------------------------------------------------------
%	ASSIGNMENT CONTENT
%----------------------------------------------------------------------------------------

\section*{Problem}

\begin{problem}
    Given a regular curve $\α : I \subset \R \to \R^3$ in $\R^3$,
its Frenet's family for some $t$ is the family $\Pmqty{κ(t), τ(t), \T(t), \N(t), \B(t)}$,
where $κ(t)$, $τ(t)$, $\T(t)$, $\N(t)$ and $\B(t)$ are
the curvature; the torsion; and the tangent, normal and bi-normal vectors
at the point $\α(t)$.

\noindent
Find the Frenet's family for the curve $\α(t) = \Pmqty{3t-t^3 & 3t^2 & 3t+t^3}$.
\end{problem}

%----------------------------------------------------------------------------------------

\subsection*{Answer}

    We need the derivatives of the curve for our calculations.

\begin{align*}
    \α(t) &= \Pmqty{3t-t^3 & 3t^2 & 3t+t^3}, \\
    \α\dv(t) &= \Pmqty{3-3t^2 & 6t & 3+3t^2} = 3\Pmqty{1-t^2 & 2t & 1+t^2}, \\
    \α\ddv(t) &= \Pmqty{-6t & 6 & 6t} = 6\Pmqty{-t & 1 & t}, \\
    \α\dddv(t) &= \Pmqty{-6 & 0 & 6} = 6\Pmqty{-1 & 0 & 1}.
\end{align*}

    We will start by finding the Fermat's frame ($\T$, $\N$ and $\B$).
$\α\dv(t)$ is already a vector proportional to $\T$ and with the same sense.
We will find a vector proportional to $\N$ and with the same sense
by removing the tangential component from $\α\ddv(t)$.
To keep the notation simple, we will call this vector $\vb{n}$.

\begin{equation*}
    \inner{\α\ddv{t}}{\α\dv(t)} =
    6 \cdot 3 \inner{\Pmqty{-t & 1 & t}}{\Pmqty{1-t^2 & 2t & 1+t^2}} =
    18\Pmqty{3t^3 + 6t + 3t^3} = 36t(1+t^2). \\
\end{equation*}

\begin{multline*}
    \vb{n}(t) = \proy_\N (\α\ddv(t)) = \\
    \α\ddv(t) -
        \frac{\inner{\α\ddv{t}}{\α\dv{t}}}{\norm{\α\dv(t)}^2}
    \α\dv(t) =
    \α\ddv(t) - \frac{36t(1+t^2)}{18(1+t^2)^2} \α\dv(t) = \\
    \α\ddv(t) - \frac{2t}{(1+t^2)} \α\dv(t) =
    \frac{1}{1+t^2} \qty\Big[(1+t^2)\α\ddv(t) - 2t \α\dv(t)] = \\
    \frac{6}{1+t^2} \qty\Big[
        (1+t^2)\Pmqty{-t & 1 & t} - t\Pmqty{1 - t^2 & 2t & 1 + t^2}
    ] = \\
    \frac{6}{1+t^2} \Pmqty{-t(1+t^2) - t + t^3 & (1+t^2) - 2t^2 & t(1+t^2) - t - t^3} = \\
    \frac{6}{1+t^2} \Pmqty{-2t & 1 - t^2 & 0}.
\end{multline*}

    Now that we have two vectors $\α\dv(t)$ and $\vb{n}(t)$
in the same direction and sense than $\T$ and $\N$,
we can find another vector with the same sense and direction as $\B$
using the cross product.

\begin{multline*}
    \vb{b}(t) = \α\dv(t) \times \vb{n} = \\
    3\Pmqty{1-t^2 & 2t & 1+t^2} \times \frac{6}{1+t^2} \Pmqty{-2t & 1 - t^2 & 0} = \\
    \frac{18}{1+t^2} \Pmqty{1-t^2 & 2t & 1+t^2} \times \Pmqty{-2t & 1 - t^2 & 0} = \\
    \frac{18}{1+t^2} \Pmqty{-(1+t^2)(1-t^2) & (1+t^2)(-2t) & (1-t^2)^2 + 4t^2} = \\
    18 \Pmqty{t^2-1 & -2t & 1+t^2}.
\end{multline*}

    All that reamins is to normalize the ortoghonal base $\qty{\α\dv(t), \vb{n}, \vb{b}}$
to obtain $\qty{\T, \N, \B}$.

\begin{align*}
    \T(t) &=
    \frac{\α\dv(t)}{\norm{\α\dv(t)}} =
    \frac{3\Pmqty{1-t^2 & 2t & 1+t^2}}{3\sqrt{2}(1+t^2)} =
    \frac{\sqrt{2}}{2} \Pmqty{\frac{1-t^2}{1+t^2} & \frac{2t}{1+t^2} & 1}, \\
%
    \N(t) &=
    \frac{\vb{n}(t)}{\norm{\vb{n}(t)}} =
    \frac{\Pmqty{-2t & 1 - t^2 & 0}}{\norm{\Pmqty{-2t & 1 - t^2 & 0}}} =
    \frac{\Pmqty{-2t & 1 - t^2 & 0}}{1+t^2} =
    \Pmqty{\frac{-2t}{1+t^2} & \frac{1-t^2}{1+t^2} & 0}, \\
%
    \B(t) &=
    \frac{\vb{b}(t)}{\norm{\vb{b}(t)}} = \\ &
    \begin{multlined}
        \frac{\Pmqty{t^2-1 & -2t & 1+t^2}}{\norm{\Pmqty{t^2-1 & -2t & 1+t^2}}} =
        \frac{\Pmqty{t^2-1 & -2t & 1+t^2}}{\frac{1}{3}\norm{\α\dv(t)}} =
        \frac{\Pmqty{t^2-1 & -2t & 1+t^2}}{\sqrt{2}(1+t^2)} = \\
        \frac{\sqrt{2}}{2}\Pmqty{\frac{t^2-1}{1+t^2} & \frac{-2t}{1+t^2} & 1}.
    \end{multlined}
\end{align*}

    In the last equality, we have writen
$\norm{\Pmqty{t^2-1 & -2t & 1+t^2}} = \frac{1}{3}\norm{\α\dv(t)}$
recognizing the similarity in the coordinates of both vectors,
and we will use $\norm{\vb{b}} = 6\norm{\α\dv(t)}$ whenever it comes in handy.
This is just a coincidence, and not an identity for any curve.

    Lastly, we will use the formulas for the curvature and the torsion.
We will try to make use of the operations that we have already made.
In particular, the relation
$\norm{\α\dv(t) \times \α\ddv(t)} = \norm{\α\dv(t) \times \vb{n}(t)}$
is a useful identity.

\begin{align*}
    κ_{\α}(t) =
    \frac{\norm{\α\dv(t) \times \α\ddv(t)}}{\norm{\α\dv(t)}^3} =
    \frac{\norm{\α\dv(t) \times \proy_\N (\α\ddv(t))}}{\norm{\α\dv(t)}^3} =
    \frac{\norm{\α\dv(t) \times \vb{n}(t)}}{\norm{\α\dv(t)}^3} = \\
    \frac{\norm{\vb{b}(t)}}{\norm{\α\dv(t)}^3} = 
    \frac{6\norm{\α\dv(t)}}{\norm{\α\dv(t)}^3} =
    \frac{6}{\norm{\α\dv(t)}^2} =
    \frac{6}{18(1+t^2)^2} =
    \frac{1}{3(1+t^2)^2}.
\end{align*}

\begin{align*}
    τ_{\α}(t) =
    -\frac{\det\Pmqty{\α\dv(t) & \α\ddv(t) & \α\dddv(t)}}
        {\norm{\α\dv(t) \times \α\ddv(t)}^2} =
    -\frac{\qty(\α\dv(t) \times \α\ddv(t)) \cdot \α\dddv(t)}
        {\norm{\α\dv(t) \times \α\ddv(t)}^2} = \\
    -\frac{\vb{b}(t) \cdot \α\dddv(t)}
        {\qty(6\norm{\α\dv(t)})^2} =
    -\frac{18 \Pmqty{t^2-1 & -2t & 1+t^2} \cdot 6\Pmqty{-1 & 0 & 1}}
        {36\cdot 18(1+t^2)^2} = \\
    -\frac{1}{3(1+t^2)^2}.
\end{align*}

%----------------------------------------------------------------------------------------

\end{document}
