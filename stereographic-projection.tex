%%%%%%%%%%%%%%%%%%%%%%%%%%%%%%%%%%%%%%%%%
% fphw Assignment
% LaTeX Template
% Version 1.0 (27/04/2019)
%
% This template originates from:
% https://www.LaTeXTemplates.com
%
% Authors:
% Class by Felipe Portales-Oliva (f.portales.oliva@gmail.com) with template 
% content and modifications by Vel (vel@LaTeXTemplates.com)
%
% Template (this file) License:
% CC BY-NC-SA 3.0 (http://creativecommons.org/licenses/by-nc-sa/3.0/)
%
%%%%%%%%%%%%%%%%%%%%%%%%%%%%%%%%%%%%%%%%%

%----------------------------------------------------------------------------------------
%    PACKAGES AND OTHER DOCUMENT CONFIGURATIONS
%----------------------------------------------------------------------------------------

\documentclass[
    12pt, % Default font size, values between 10pt-12pt are allowed
    %letterpaper, % Uncomment for US letter paper size
    %spanish, % Uncomment for Spanish
]{fphw}

% Template-specific packages
\usepackage[utf8]{inputenc} % Required for inputting international characters
\usepackage[T1]{fontenc} % Output font encoding for international characters
\usepackage{fontspec,unicode-math} % Required for using utf8 characters in math mode
\usepackage{parskip}  % To add extra space between paragraphs
% \usepackage{mathpazo} % Use the Palatino font
\usepackage{graphicx} % Required for including images
\usepackage{booktabs} % Better horizontal rules in tables
\usepackage{hyperref} % For links (both internal and external)
% \usepackage{listings} % Required for insertion of code
\usepackage{enumerate}% To modify the enumerate environment
\usepackage{cleveref} % Better \ref command -> \cref
\usepackage{import}   % This 4 packages and the command allow importing pdf
\usepackage{xifthen}  % figures generated with inkscape
\usepackage{pdfpages} % Source: https://castel.dev/post/lecture-notes-2/
\usepackage{mathtools}
\usepackage{wrapfig}
\usepackage{cancel}
\usepackage{transparent}
\newcommand{\incfig}[1]{%
    \def\svgwidth{0.95\columnwidth}
    \small
        \import{./images/}{#1.pdf_tex}
}

\setlength{\parindent}{15pt}
\setlength{\headheight}{22.66pt}

%----------------------------------------------------------------------------------------
%    ASSIGNMENT INFORMATION
%----------------------------------------------------------------------------------------

\title{Task 5 \\ Stereographic Projection} % Assignment title

\author{Emilio Domínguez Sánchez} % Student name

\date{November 4th, 2020} % Due date

\institute{University of Murcia \\ Faculty of Mathematics} % Institute or school name

\class{Geometría de Superficies} % Course or class name

\professor{Dr. Pascual Lucas Saorin} % Professor or teacher in charge of the assignment

%----------------------------------------------------------------------------------------
%    Definitions
%----------------------------------------------------------------------------------------

\usepackage{physics}
\newcommand{\R}{\mathbb{R}}
\newcommand{\sphere}{{\mathbb{S}^2}}
\newcommand{\inner}[2]{\left\langle #1, \; #2 \right\rangle}

\begin{document}

\maketitle % Output the assignment title, created automatically using the information in the custom commands above

%----------------------------------------------------------------------------------------
%    ASSIGNMENT CONTENT
%----------------------------------------------------------------------------------------

\section*{Problem}

\begin{problem}

The stereographic projection is a function that projects the sphere given by the equation

\begin{equation*}
    \sphere = \qty{\mqty(x & y & z) : x^2 + y^2 + (z-1)^2 = 1}
\end{equation*}

onto the plane $z = 0$, which we identify with $\R^2$.
The projection; $π$; maps the point $p = \mqty(x & y & z) \in \sphere - \qty{N}$,
where $N = \mqty(0 & 0 & 2)$,
to the insersection of the plane $z = 0$ with the line that connects $N$ and $p$.

\begin{enumerate}
    \item Prove that $X = π^{-1} : \R^2 \to \sphere - \qty{N}$
    is a parametrization of the sphere.

    \item Find the coefficients of the first fundamental form of $X$.
\end{enumerate}
\end{problem}

%----------------------------------------------------------------------------------------

\subsection*{Answer}

    The statement talks about $π$ and gives a valid definition for it because
the intersection of a line and a plane is at most one point and
the lines that join $N$ and a different point are never parallel to the plane $z=0$.
However, all the questions are related to $X ≔ π^{-1}$.
Out approach finds the expression for $X$ without computing the expression of $π$.
Nevertheless, we can use the fact that we know that $π$ is well defined
to define $X$ in a way that it is $π^{-1}$ (by construction)
and is therefore biyective (hence inyective),
which comes in handy to prove that it is a parametrization
without proving that the inverse is also continuos.
That is, without finding an expression for $π$.

    Take a point $\mqty(u & v)$ of $\R^2$.
$X(u, v)$ is the image of a point $\mqty(x & y & z)$ if and only if
$\mqty(u & v & 0)$ is the intersection of the plane $z = 0$ and
the line $l ≔ \qty{N + λ(\mqty(x & y & z)-N)}_{λ \in \R}$.
Hence, $l$ is also the line that passes through $N$ and $\mqty(u & v & 0)$,
and $π^{-1}(\qty{\mqty(u & v)})$ is, by definition, $l \cap (\sphere - N)$.

\begin{equation*}
    l = 
    \qty{\mqty(u & v & 0) + λ\qty(N-\mqty(u & v & 0))}_{λ \in \R} =
    \qty{\mqty(u(1-λ) & v(1-λ) & 2λ)}_{λ \in \R}. \\
\end{equation*}
\begin{equation*}
    l \cap \sphere = \\
    \qty{\mqty(u(1-λ) & v(1-λ) & 2λ)}_{λ \in \R} \cap \qty{x^2 + y^2 + (z-1)^2 = 1},
\end{equation*}

\pagebreak

    Substituting the parametric equations of $l$ into the implicit equation of $\sphere$
leaves us with the equation\footnote{
Note that we know $l(1) = N \in \sphere$,
which means that $λ = 1$ is a solution of the equation.}

\begin{gather*}
    \begin{aligned}
        u^2(1-λ)^2 + v^2(1-λ)^2 + (2λ-1)^2 &= 1 \\
        (1-λ)\qty[(u^2+v^2)(1-λ) - 4λ] &= 0,
    \end{aligned} \\
\end{gather*}

\noindent
of solutions $λ = 1$ and $λ = \frac{u^2+v^2}{u^2+v^2+4} = 1 - \frac{4}{u^2+v^2+4}$.
That gives

\begin{align*}
    l \cap \sphere &=
    \qty{
        \mqty(\frac{4u}{u^2+v^2+4} & \frac{4v}{u^2+v^2+4} & 2-\frac{8}{u^2+v^2+4}),
        N
    } \qq{and} \\
    l \cap (\sphere - N) &=
    \qty{
        \mqty(\frac{4u}{u^2+v^2+4} & \frac{4v}{u^2+v^2+4} & 2-\frac{8}{u^2+v^2+4})
    }.
\end{align*}

    The fact that the intersection is defined for all $\mqty(u & v)$ and is unique
proves that $π$ is a biyection between $\sphere - N$ and $\R^2$.
And the expression for $π^{-1}$ would be

\begin{equation*}
    X(u, v) =
    \mqty(\dfrac{4u}{u^2+v^2+4} & \dfrac{4v}{u^2+v^2+4} & 2-\dfrac{8}{u^2+v^2+4}).
\end{equation*}

    Now we foucs on the properties of $X$. Its differential is

\begin{equation*}
    %TODO \mqty was not handled as I expected in a subindex
    \dd{X}_{(u \; v)} =
    \pmqty{
    -4u^2 + 4v^2 + 16 & -8uv \\
    -8uv & 4u^2 - 4v^2 + 16 \\
    16u & 16v \\
    } \times \frac{1}{(u^2+v^2+4)^2}
\end{equation*}

\noindent
which is of maximum range because

\begin{multline*}
    \mqty|
    -4u^2 + 4v^2 + 16 & -8uv \\
    -8uv & 4u^2 - 4v^2 + 16
    | = \\
    16^2 - (4u^2-4v^2)^2 - (8uv)^2 =
    16^2 - 16(u^4+v^4) + 32u^2v^2 - 64u^2v^2 = \\
    16^2 - (4(u^2+v^2))^2,
\end{multline*}
\begin{multline*}
    \mqty|
    -4u^2 + 4v^2 + 16 & -8uv \\
    16u & 16v \\
    | = \\
    -64u^2v + 64v^3 + 256v + 128u^2v =
    v(64(u^2 + v^2) + 256)
\end{multline*}
\noindent
and
\begin{multline*}
    \mqty|
    -8uv & 4u^2 - 4v^2 + 16 \\
    16u & 16v \\
    | = \\
    -128uv^2 - 64u^3 + 64uv^2 - 256u =
    u(-64(u^2 + v^2) - 256)
\end{multline*}

\noindent
are $0$ when $u^2+v^2 = 4$, $v = 0$ and $u = 0$, respectively,
which cannot happen at the same time.
All the work up until now,
plus the fact that $X$ in inyective because it is the inverse of $π$,
is enough to apply the technical result that proves that
$X^{-1}$ exists and is continuous.
In other words, to show that $X$ is a parametrization of $\sphere$.

\subsubsection*{Coefficients}

\newcommand{\dX}[1]{{\pdv{X_{(u \; v)}}{#1}}}

Writing

\begin{align*}
    \dX{u} &= \frac{\mqty(-4u^2 + 4v^2 + 16 & -8uv & 16u)}{(u^2+v^2+4)^2}, \\
    \dX{v} &= \frac{\mqty(-8uv & 4u^2 - 4v^2 + 16 & 16v)}{(u^2+v^2+4)^2};
\end{align*}

\noindent
we get that the coefficients are

\begin{multline*}
    {E = \inner{\dX{u}}{\dX{u}} = } \\
    \frac{1}{(u^2+v^2+4)^4}
        \qty(16u^4+16v^4+256-32u^2v^2-128u^2+128v^2 + 64u^2v^2 + 256u^2) = \\
    \frac{1}{(u^2+v^2+4)^4}
        \qty(16u^4+16v^4 +32u^2v^2 + 128u^2 + 128v^2 + 256) = \\
    \frac{16}{(u^2+v^2+4)^4}
        \qty(u^4+v^4 +2u^2v^2 + 8u^2 + 8v^2 + 16) =
    \frac{16}{(u^2+v^2+4)^2},
\end{multline*}
\begin{multline*}
    {G = \inner{\dX{v}}{\dX{v}} = }
    \hfill \text{(by symmetry)} \hfill =
    \frac{16}{(u^2+v^2+4)^2}
\end{multline*}
and
\begin{multline*}
    {F = \inner{\dX{u}}{\dX{v}} = }
    \hfill \frac{1}{(u^2+v^2+4)^4} \qty(-128uv - 128uv + 256uv) \hfill =
    0.
\end{multline*}

(Note that $X$ is an orthogonal parametrization because $F = 0$.)

%----------------------------------------------------------------------------------------

\end{document}
