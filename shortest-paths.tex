%%%%%%%%%%%%%%%%%%%%%%%%%%%%%%%%%%%%%%%%%
% fphw Assignment
% LaTeX Template
% Version 1.0 (27/04/2019)
%
% This template originates from:
% https://www.LaTeXTemplates.com
%
% Authors:
% Class by Felipe Portales-Oliva (f.portales.oliva@gmail.com) with template 
% content and modifications by Vel (vel@LaTeXTemplates.com)
%
% Template (this file) License:
% CC BY-NC-SA 3.0 (http://creativecommons.org/licenses/by-nc-sa/3.0/)
%
%%%%%%%%%%%%%%%%%%%%%%%%%%%%%%%%%%%%%%%%%

%----------------------------------------------------------------------------------------
%	PACKAGES AND OTHER DOCUMENT CONFIGURATIONS
%----------------------------------------------------------------------------------------

\documentclass[
	12pt, % Default font size, values between 10pt-12pt are allowed
	%letterpaper, % Uncomment for US letter paper size
	%spanish, % Uncomment for Spanish
]{fphw}

% Template-specific packages
\usepackage[utf8]{inputenc} % Required for inputting international characters
\usepackage[T1]{fontenc} % Output font encoding for international characters
\usepackage{fontspec,unicode-math} % Required for using utf8 characters in math mode
\usepackage{parskip}  % To add extra space between paragraphs
\usepackage{graphicx} % Required for including images
\usepackage{booktabs} % Required for better horizontal rules in tables
% \usepackage{listings} % Required for insertion of code
\usepackage{enumerate}% To modify the enumerate environment
\usepackage{import}   % This 4 packages and the command allow importing pdf
\usepackage{xifthen}  % figures generated with inkscape
\usepackage{pdfpages} % Source: https://castel.dev/post/lecture-notes-2/
\usepackage{transparent}
\newcommand{\incfig}[1]{%
    \def\svgwidth{0.45\columnwidth}
    \small
        \import{./images/}{#1.pdf_tex}
}
\setlength{\parindent}{15pt}
\setlength{\headheight}{22.66pt}

%----------------------------------------------------------------------------------------
%	ASSIGNMENT INFORMATION
%----------------------------------------------------------------------------------------

\title{Task 4 \\ Shortest paths} % Assignment title

\author{Emilio Domínguez Sánchez} % Student name

\date{May 7th, 2021} % Due date

\institute{University of Murcia \\ Faculty of Mathematics} % Institute or school name

\class{Geometría Global de Superficies} % Course or class name

\professor{Dr. Luis J. Alías Linares} % Professor or teacher in charge of the assignment

%----------------------------------------------------------------------------------------
%	Definitions
%----------------------------------------------------------------------------------------

\setmathfont{latinmodern-math.otf}
\setmathfont[range=\setminus]{asana-math.otf}
\usepackage{physics}
\usepackage{cleveref}
\usepackage{amsthm}
\newtheorem{lemma}{Lemma}
\newcommand{\R}{\mathbb{R}}
\newcommand{\clsr}[1]{\overline{#1}}
\DeclareMathOperator{\len}{len}

\begin{document}

\maketitle % Output the assignment title, created automatically using the information in the custom commands above

%----------------------------------------------------------------------------------------
%	ASSIGNMENT CONTENT
%----------------------------------------------------------------------------------------

\section*{Problem}

\begin{problem}
    Let $S$ be a regular surface and
    $V$ a normal neighborhood centered in $p \in S$.
    Let $ε$ be small enough so that the closure of the geodesic disk $D(p, ε)$
    is contained in $V$, $\clsr{D}(p, ε) \subset V$.

    Let $q \in S \setminus D(p, ε)$ and consider the function $r(x) := d(x, q)$
    that measures the distance along $S$ from $x$ to $q$.
    Let $m$ be a point where the function $r$ reaches a minimum
    over the compact $S(p, ε)$.
    $m := \min_{x \in S(p, ε)} \{r(x)\}$.
    Then
    \begin{equation*}
        d(p, q) = d(p, m) + d(m, q) = ε + d(m, q).
    \end{equation*}
\end{problem}

%----------------------------------------------------------------------------------------

The length minimizing property of the geodesics tells us that if $D(p,r)$
is contained in a normal neighborhood of $p$,
then the geodesic that joins $p$ with $x \in D(p,r)$ is a shortest path.
Because $\clsr{D}(p,ε) \subset V$, we can find $r > ε$ such that $D(p,r) \subset V$.
Applying the length minimizing property to $S(p,ε) \subset D(p,r)$,
we conclude that $d(p,x) = ε$ for all $x \in S(p,ε)$.

Once we prove that, the rest of the exercise involves reasonings
common to any distance $d(p,q)$ defined as
the minimum length of the paths joining $p$ and $q$\footnote{
    For some notions of length and path that follow the common sense of
    what is a path and what is a length.
}.
Here I have chosen to break the proof in small results.

\begin{lemma}
    Fix a path $α : [0, l]$ from $p$ to $q$.
    Given that $p$ belongs to $D(p,ε)$, but $q$ does not,
    $α$ needs to traverse the boundary of $D(p,ε)$.
    That is, $α([0, l])) \cap S(p,ε) \ne \emptyset$.
\end{lemma}

\begin{proof}
    Let $U_1 := α^{-1}(D(p,ε))$ and $U_2 := α^{-1}(S \setminus \clsr{D}(p,ε))$.
    Both are the preimage of an open set and are therefore open.
    If the intersection $α([0,l]) \cap S(p,ε)$ were empty, then $U_1 \cup U_2$
    would be a separation of $[0,l]$ in two open disjoint sets, which is absurd.
\end{proof}

\begin{lemma}
    If every path from $p$ to $q$ needs to traverse a set $I$, then
    \begin{equation}\label{eq:in}
        d(p,q) \ge d(p,I) + d(I,q).
    \end{equation}
\end{lemma}

\begin{proof}
    Let $α : [0,l] \to S$ be any path from $p$ to $q$ and
    let $t$ be such that $α(t) \in I$,
    which exists because $α$ needs to traverse $I$.
    Then $\len_0^l(α) = \len_0^{t}(α) + \len_{t}^l(α) \ge d(p, I) + d(I, q)$.
\end{proof}

\begin{figure}[hbt]
    \centering
    \hspace{0.15\columnwidth}
    \incfig{distances}
    \caption{An example of a set that separates two points and where
    the inequality \eqref{eq:in} is strict.}
    \label{fig:in}
\end{figure}

Notice that the inequality \eqref{eq:in} may be strict (\cref{fig:in}),
because the points $x \in I$ for which $d(p, x) = d(p, I)$, if they exist,
can be different from those where $d(x, q) = d(I, q)$.
In our case though, $d(p, x) = ε$ for all $x \in S(p,ε)$.

We conclude the proof writing
\begin{equation*}
    d(p, m) + d(m, q) = ε + d(m, q) = d(p, S(p,ε)) + d(S(p,ε), q) \le d(p, q).
\end{equation*}

%----------------------------------------------------------------------------------------

\end{document}
