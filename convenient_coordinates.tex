%%%%%%%%%%%%%%%%%%%%%%%%%%%%%%%%%%%%%%%%%
% fphw Assignment
% LaTeX Template
% Version 1.0 (27/04/2019)
%
% This template originates from:
% https://www.LaTeXTemplates.com
%
% Authors:
% Class by Felipe Portales-Oliva (f.portales.oliva@gmail.com) with template 
% content and modifications by Vel (vel@LaTeXTemplates.com)
%
% Template (this file) License:
% CC BY-NC-SA 3.0 (http://creativecommons.org/licenses/by-nc-sa/3.0/)
%
%%%%%%%%%%%%%%%%%%%%%%%%%%%%%%%%%%%%%%%%%

%----------------------------------------------------------------------------------------
%    PACKAGES AND OTHER DOCUMENT CONFIGURATIONS
%----------------------------------------------------------------------------------------

\documentclass[
    12pt, % Default font size, values between 10pt-12pt are allowed
    %letterpaper, % Uncomment for US letter paper size
    %spanish, % Uncomment for Spanish
]{fphw}

% Template-specific packages
\usepackage[utf8]{inputenc} % Required for inputting international characters
\usepackage[T1]{fontenc} % Output font encoding for international characters
\usepackage{fontspec,unicode-math} % Required for using utf8 characters in math mode
\usepackage{parskip}  % To add extra space between paragraphs
% \usepackage{mathpazo} % Use the Palatino font
\usepackage{graphicx} % Required for including images
\usepackage{booktabs} % Better horizontal rules in tables
\usepackage{hyperref} % For links (both internal and external)
% \usepackage{listings} % Required for insertion of code
\usepackage{enumerate}% To modify the enumerate environment
\usepackage{cleveref} % Better \ref command -> \cref
\usepackage{import}   % This 4 packages and the command allow importing pdf
\usepackage{xifthen}  % figures generated with inkscape
\usepackage{pdfpages} % Source: https://castel.dev/post/lecture-notes-2/
\usepackage{mathtools}
\usepackage{wrapfig}
\usepackage{cancel}
\usepackage{transparent}
\newcommand{\incfig}[1]{%
    \def\svgwidth{0.95\columnwidth}
    \small
        \import{./images/}{#1.pdf_tex}
}

\setlength{\parindent}{15pt}
\setlength{\headheight}{22.66pt}

%----------------------------------------------------------------------------------------
%    ASSIGNMENT INFORMATION
%----------------------------------------------------------------------------------------

\title{Task 3 \\ Convenient Coordinates} % Assignment title

\author{Emilio Domínguez Sánchez} % Student name

\date{March 16th, 2021} % Due date

\institute{University of Murcia \\ Faculty of Mathematics} % Institute or school name

\class{Geometría Global de Superficies} % Course or class name

\professor{Dr. Luis J. Alías Linares} % Professor or teacher in charge of the assignment

%----------------------------------------------------------------------------------------
%    Definitions
%----------------------------------------------------------------------------------------

\usepackage{physics}
\newcommand{\R}{\mathbb{R}}
\newcommand{\inner}[2]{\left\langle #1, \; #2 \right\rangle}

\DeclareDocumentCommand\covariantderivative{ s o m g d() }
{ % Covariant derivative
	% s: star for \flatfrac flat covariant derivative
	% o: optional n for nth covariant derivative
	% m: mandatory (x in Df/dx)
	% g: optional (f in Df/dx)
	% d: long-form D/dx(...)
	\IfBooleanTF{#1}
	{\let\fractype\flatfrac}
	{\let\fractype\frac}
	\IfNoValueTF{#4}
	{
		\IfNoValueTF{#5}
		{\fractype{D \IfNoValueTF{#2}{}{^{#2}}}{\diffd #3\IfNoValueTF{#2}{}{^{#2}}}}
		{\fractype{D \IfNoValueTF{#2}{}{^{#2}}}{\diffd #3\IfNoValueTF{#2}{}{^{#2}}} \argopen(#5\argclose)}
	}
	{\fractype{D \IfNoValueTF{#2}{}{^{#2}} #3}{\diffd #4\IfNoValueTF{#2}{}{^{#2}}}}
}
\DeclareDocumentCommand\cdv{}{\covariantderivative} % Shorthand for \covariantderivative

\begin{document}

\maketitle % Output the assignment title, created automatically using the information in the custom commands above

%----------------------------------------------------------------------------------------
%    ASSIGNMENT CONTENT
%----------------------------------------------------------------------------------------

\section*{Problem}

\begin{problem}
    Let $p \in S$ be a point of a regular surface $S$
    and $\qty{e_1, e_2}$ an orthonormal basis of $T_pS$.
    Consider $X(u, v) = \exp_p(ue_1 + ve_2)$,
    a normal coordinate system centered in $p$
    that is defined over a neighbourhood $U$ of $(0, 0) \in \R^2$.
    Prove that the following statements are true.

    \begin{enumerate}
        \item For any $v = ae_1 + be_2 \in T_pS$, $v \ne 0$,
        the coordinate expression of the maximal geodesic $γ_v(t)$ is
        $(u(t), v(t)) = (at, bt)$.

        \item The Christoffel symbols $\qty{Γ_{ij}^k : 1 \le i,j,k \le 3}$
        and the partial derivatives of $E$, $F$ and $G$ are zero in $(0, 0)$.
    \end{enumerate}
\end{problem}

%----------------------------------------------------------------------------------------

\subsection*{Answer}

    By the homogenity of geodesics and the definition of the exponential function,
$γ_v(t) = γ_{tv}(1) = \exp_p(tv) = \exp_p(tae_1 + tbe_2)$.
Because $X$ has to be inyective,
$(u(t), v(t)) = t(a, b)$ is the coordinates expression of $γ_v$.
Thanks to this simple expression,
the differential equation for a geodesic\footnote{
    For simplicity,
    we have omitted the point of evaluation $(u, v)$ of the Christoffel symbols.
},
\begin{equation*}
    \begin{cases}
        u'' + (u')^2Γ^1_{11} + 2u'v'Γ^1_{12} + (v')^2Γ^1_{22} = 0 \\
        v'' + (u')^2Γ^2_{11} + 2u'v'Γ^2_{12} + (v')^2Γ^2_{22} = 0
    \end{cases},
\end{equation*}
become
\begin{equation*}
    \begin{cases}
        a^2Γ^1_{11} + 2abΓ^1_{12} + b^2Γ^1_{22} = 0 \\
        a^2Γ^2_{11} + 2abΓ^2_{12} + b^2Γ^2_{22} = 0
    \end{cases}.
\end{equation*}
Given that $v$ is arbitrary and that
$X(0, 0) = p = γ_v(0)$ is a point common to any $γ_v$,
we can conclude that all the Christoffel symbols are $0$ in $(0, 0)$.
That is,
substituting with $v = e_1$ gives $Γ^1_{11}(0, 0), Γ^2_{11}(0, 0) = 0$,
substituting with $v = e_2$ gives $Γ^1_{22}(0, 0), Γ^2_{22}(0, 0) = 0$,
and knowing that,
substituting with $v = e_1 + e_2$ gives $Γ^1_{12}(0, 0), Γ^2_{12}(0, 0) = 0$.
Ultimately, writing
\begin{equation*}
    \mqty(
        \frac{E_u}{2} & \frac{E_v}{2} & F_v - \frac{G_u}{2} \\
        F_u - \frac{E_v}{2} & \frac{G_u}{2} & \frac{G_v}{2}
    ) =
    \mqty(E & F \\ F & G)
    \mqty(Γ^1_{11} & Γ^1_{12} & Γ^1_{22} \\ Γ^2_{11} & Γ^2_{12} & Γ^2_{22}) = 0,
\end{equation*}
we can conclude that the partials of $E$, $F$ and $G$ are $0$ in $(0, 0)$ as well.

%----------------------------------------------------------------------------------------

\end{document}
