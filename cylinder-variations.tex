%%%%%%%%%%%%%%%%%%%%%%%%%%%%%%%%%%%%%%%%%
% fphw Assignment
% LaTeX Template
% Version 1.0 (27/04/2019)
%
% This template originates from:
% https://www.LaTeXTemplates.com
%
% Authors:
% Class by Felipe Portales-Oliva (f.portales.oliva@gmail.com) with template 
% content and modifications by Vel (vel@LaTeXTemplates.com)
%
% Template (this file) License:
% CC BY-NC-SA 3.0 (http://creativecommons.org/licenses/by-nc-sa/3.0/)
%
%%%%%%%%%%%%%%%%%%%%%%%%%%%%%%%%%%%%%%%%%

%----------------------------------------------------------------------------------------
%	PACKAGES AND OTHER DOCUMENT CONFIGURATIONS
%----------------------------------------------------------------------------------------

\documentclass[
	12pt, % Default font size, values between 10pt-12pt are allowed
	%letterpaper, % Uncomment for US letter paper size
	%spanish, % Uncomment for Spanish
]{fphw}

% Template-specific packages
\usepackage[utf8]{inputenc} % Required for inputting international characters
\usepackage[T1]{fontenc} % Output font encoding for international characters
\usepackage{fontspec,unicode-math} % Required for using utf8 characters in math mode
\usepackage{parskip}  % To add extra space between paragraphs
\usepackage{graphicx} % Required for including images
\usepackage{booktabs} % Required for better horizontal rules in tables
% \usepackage{listings} % Required for insertion of code
\usepackage{enumerate}% To modify the enumerate environment
\usepackage{import}   % This 4 packages and the command allow importing pdf
\usepackage{xifthen}  % figures generated with inkscape
\usepackage{pdfpages} % Source: https://castel.dev/post/lecture-notes-2/
\usepackage{transparent}
\newcommand{\incfig}[1]{%
    \def\svgwidth{0.45\columnwidth}
    \small
        \import{./images/}{#1.pdf_tex}
}
\setlength{\parindent}{15pt}
\setlength{\headheight}{22.66pt}

%----------------------------------------------------------------------------------------
%	ASSIGNMENT INFORMATION
%----------------------------------------------------------------------------------------

\title{Task 5 \\ Cylinder Variations} % Assignment title

\author{Emilio Domínguez Sánchez} % Student name

\date{May 9th, 2021} % Due date

\institute{University of Murcia \\ Faculty of Mathematics} % Institute or school name

\class{Geometría Global de Superficies} % Course or class name

\professor{Dr. Luis J. Alías Linares} % Professor or teacher in charge of the assignment

%----------------------------------------------------------------------------------------
%	Definitions
%----------------------------------------------------------------------------------------

\setmathfont{latinmodern-math.otf}
\setmathfont[range=\setminus]{asana-math.otf}
\usepackage{physics}
\usepackage{cleveref}
\usepackage{amsthm}
\newtheorem{lemma}{Lemma}
\newcommand{\R}{\mathbb{R}}
\newcommand{\clsr}[1]{\overline{#1}}
\newcommand{\inner}[2]{\left\langle #1, \; #2 \right\rangle}
\DeclareMathOperator{\len}{len}

\begin{document}

\maketitle % Output the assignment title, created automatically using the information in the custom commands above

%----------------------------------------------------------------------------------------
%	ASSIGNMENT CONTENT
%----------------------------------------------------------------------------------------

\section*{Problem}

\begin{problem}
    Consider a regular curve $α:[a,b] \to \R^3$ parametrized by arc
    of the form $α(t) = \mqty(α_1(t) & α_2(t) & 0)$.
    Let $S \subset \R^3$ be the cylinder built over $α$,
    given by its natural parametrization $X : [a,b] \times \R \to S$
    \begin{equation*}
        X(s,v) = α(s) + \mqty(0 & 0 & v) = \mqty(α_1(s) & α_2(s) & v).
    \end{equation*}

    \begin{enumerate}
        \item Let $f : [a,b] \times (-ε,ε) \to \R$ be a differentiable function.
        Prove that the application $φ : [a,b] \times (-ε,ε) \to S$
        \begin{equation*}
            φ(s, t) = α(s) + \mqty(0 & 0 & f(s, t)) = \mqty(α_1(s) & α_2(s) & f(s, t))
        \end{equation*}
        defines a proper normal variation of $α$ if and only if
        the function $f$ satisfies $f(s,0) = f(a,t) = f(b,t) = 0$
        for all $s \in [a,b]$ and $t \in (-ε,ε)$.

        \item In the conditions of the previous item,
        when $φ$ is a proper normal variation,
        let $L(t)$ be the length of the curve $α_t(s) = φ(s, t)$,
        \begin{equation*}
            L(t) = \int_a^b \norm{α_t'(s)} \dd{s}.
        \end{equation*}
        Find $L'(0)$ and $L''(0)$.
    \end{enumerate}
\end{problem}

%----------------------------------------------------------------------------------------

We start by writing the properties of a normal proper variation.
\begin{itemize}
    \item $φ$ is differentiable because it is the sum of two differentiable functions.
    Therefore, it is continous.
    \item $α_t$ is a regular curve because $φ$ is differentiable and
    \begin{equation*}
        \norm{α_t'(s)}^2 = \norm{α'(s)}^2 + \norm{\mqty(0 & 0 & f'(s, t))}^2
            \ge \norm{α'(s)}^2.
    \end{equation*}
    \item $α_0(s) = α(s) + f(s,0)e_3 = α(s)$ if and only if $f(s,0) = 0$.
    \item $α_t(a) = α(a) + f(a,t)e_3 = α(a)$ if and only if $f(a,t) = 0$.
    \item $α_t(b) = α(b) + f(b,t)e_3 = α(b)$ if and only if $f(b,t) = 0$.
    \item $φ_t(s,0) = f_t(s,0)e_3$ is always normal to $α'(s)$.
\end{itemize}
This concludes the first part of the exercise.

%----------------------------------------------------------------------------------------

From now on, we suppose the conditions hold and
we start the study of the length function.
Applying the Leibniz integral rule to differentiate the integral,
\begin{equation*}
    L'(t) = \int_a^b \pdv{t} \norm{a_t'(s)} \dd{s} =
    \int_a^b \pdv{t} \norm{φ_s(s,t)} \dd{s} =
    \int_a^b \frac{\inner{φ_{st}(s,t)}{φ_s(s,t)}}{\norm{φ_s(s,t)}} \dd{s}.
\end{equation*}
In this case, $φ_{st} = f_{st}e_3$, $φ_s(s,t) = α'(s) + f_s(s,t)e_3$,
and the inner product is
\begin{equation*}
    \inner{φ_{st}(s,t)}{φ_s(s,t)} = f_{st}(s,t)f_s(s,t).
\end{equation*}
Hence, the length derivative can be computed as
\begin{equation}\label{eq:dv}
    L'(t) = \int_a^b \frac{f_{st}(s,t)f_s(s,t)}{\norm{φ_s(s,t)}} \dd{s},
\end{equation}
but $\eval{f_s(s, t)}_{t = 0} = \pdv{s} f(s, 0) = 0$,
and so $L'(0) = 0$.

For the second derivative, we use Leibniz integral rule once more
for the equation \eqref{eq:dv}.
\begin{align*}
    L''(t) & = \int_a^b \pdv{t} \frac{f_{st}(s,t)f_s(s,t)}{\norm{φ_s(s,t)}} \dd{s} \\
    & = \int_a^b \qty\Bigg[
        \frac{f_{stt}(s,t)f_s(s,t) + f_{st}(s,t)^2}{\norm{φ_s(s,t)}}
        - \frac{\qty[f_{st}(s,t)f_s(s,t)]^2}{\norm{φ_s(s,t)}^2}
    ] \dd{s}
\end{align*}
Using that $\eval{φ_s(s,t)}_{t = 0} = 0$ once more, the second derivative at $t = 0$ is
\begin{align*}
    L''(0) = \int_a^b \frac{f_{st}(s,0)^2}{\norm{φ_s(s,0)}} \dd{s}.
\end{align*}
Finally, we can clean the formula a little bit more using that
$φ_s(s, 0) = \dv{s} φ(s, 0) = \dv{s} α(s) = α'(s)$
and that $α$ is parametrized by arc.
\begin{equation*}
    L''(0) = \int_a^b f_{st}(s,0)^2 \dd{s}
\end{equation*}

% ALTENATE FORM
% But we are interested in $L'(0)$.
% In particular, $\eval{φ_s(s,t)}_{t = 0} = \pdv{s} α(s) = α'(s)$.
% Using that $α$ is parametrized by arc and the change
% $\inner{φ_{st}}{φ_s} = \pdv{s} \inner{φ_t}{φ_s} - \inner{φ_t}{φ_{ss}}$,
% we can further simplify the expression as
% \begin{align*}
%     L'(0) & = \int_a^b \qty\Bigg[
%         \dv{s} \inner{φ_t(s,0)}{φ_s(s,0)} - \inner{φ_t(s,0)}{φ_{ss}(s,0)}
%     ] \dd{s} \\
%     & = \eval(\inner{φ_t(s,0)}{φ_s(s,0)}|_a^b
%         - \int_a^b \inner{φ_t(s,0)}{φ_{ss}(s,0)} \dd{s} \\
% \end{align*}
% The first part is $0$ because $φ$ is a proper variation:
% $φ_t(a,t) = \dv{t} α(a) = 0 = φ_t(b,t) = \dv{t} α(b) = 0$.
% The second part is also $0$ because
% $φ_t(s,0) = f_t(s,0)e_3$ and $φ_{ss}(s,0) = \dv[2]{s}α(s) \perp e_3$.
% Therefore $L'(0) = 0$.

%----------------------------------------------------------------------------------------

\end{document}
